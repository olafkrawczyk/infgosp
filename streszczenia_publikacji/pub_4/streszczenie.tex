\documentclass{article}

\usepackage[utf8]{inputenc}
\usepackage{polski}
\usepackage{indentfirst}

\begin{document}
\begin{center}
	\LARGE Streszczenie
\end{center}
\begin{center}
	{\Large A Study on Human Activity Recognition Using Accelerometer Data
		from Smartphones - Akram Bayat, Marc Pomplun, Duc A. Tran}
\end{center}

\section{Wstęp}
W pracy opisanej w tym dokumencie zaproponowane zostało podejście do rozpoznawania aktywności użytkownika
na podstawie danych otrzymywanych z akcelerometru wbudowanego w telefon komórkowy.

Do rozpoznawania aktywności na podstaie uzyskanych danych wykorzystano nadzorowane uczenie maszynowe.

Zbudowany przez autorów pracy model służy do rozpoznawania następujących aktywności:
\begin{itemize}
	\item powolny chód,
	\item szybki chód,
	\item bieg,
	\item wchodzenie po schodach,
	\item zchodzenie ze schodów
\end{itemize}

Dane zostały zebrane w postaci wartości x, y, z zwracanych przez akcelerometr.
Zebrane dane pochodziły od dwu kobiet oraz mężczyzn, którzy kolejno wykonywali wymienione aktywności. Czas wyknoywania aktywności wahał się od 180 do 280 sekund. Dane były zbierane z częstotliwością 100Hz.

\section{Ekstrakcja cech}

Rozpoznawanie aktywności nie przebiega na surowych danych, lecz na wektorach cech.
Dla serii danych $A_x, A_y, A_z$ wykorzystano filtr dolnoprzepustowy, aby oddzielić składowe o wysokiej częstotliwośći (AC) od niskiej (DC) powiązanych kolejno z aktywnością użytkownika oraz przyspieszeniem ziemskim. Dokument dokładniej opisuje budowę tego filtra.

Zastosowanie filtra spowodowało dodanie kolejnych 6 serii danych $A_{DCi}, A_{ACi}$.
Do serii danych dodano również wartość $A_m$, która odpowiada długości wektora [x, y, z].

Z utworzonych 24 cech (dokładny opis w dokumencie strona 5) wybrano 18:
\begin{itemize}
	\item średnia wartość na osi z,
	\item MinMax, odchylenie statndardowe, średnia kwadratowa dla $A_m$,
	\item APF - średnie liczba szczytów w oknie osie x,y,z,
	\item wariancja APF dla osi x,y,z,
	\item odchylenie standardowe dla osi x,y,z,
	\item korelacja pomiędzy osiami z oraz y,
	\item MinMax dla osi x,y,z
\end{itemize}

MinMax to różnica pomiędzy maksymalną, a minimalną wartością sygnału występującymi w oknie.

\section{Klasyfikatory}
W opisywanej tutaj pracy przebadano dokładność następujących klasyfikatorów:
\begin{itemize}
	\item Multilayer Perceptron
	\item SVM
	\item Random Forest
	\item LMT
	\item Simple Logistic
	\item Logit Boost
\end{itemize}

Przeporowadzono analizę dokładności wykrywania aktywności dla telefonu umieszczonego w kieszeni oraz trzymanego w ręce. Podczas gdy telefon był umieszczony w kieszeni uzyskano dokładność 90.34\%. Szczegółowe wyniki przedstawione są na stronie 6.
\end{document}